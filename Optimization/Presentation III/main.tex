\documentclass{beamer}
\usetheme{AnnArbor}
\usecolortheme{spruce}
\usepackage{amsmath}
\usepackage{amssymb}
\usepackage{graphicx}
\usepackage{booktabs}
\usepackage{multirow}
\usepackage{ragged2e} 


\setbeamertemplate{headline}
{
  \leavevmode%
  \hbox{%
  \begin{beamercolorbox}[wd=\paperwidth,ht=3.5ex,dp=1.5ex]{section in head/foot}%
    \usebeamerfont{section in head/foot}%
    \fontsize{8pt}{10pt}\selectfont%
    \insertsectionnavigationhorizontal{\paperwidth}{}{\hskip0pt plus1fill}{\hskip0pt plus1filll}
  \end{beamercolorbox}%
  }
}

% Custom footer
\setbeamertemplate{footline}
{
  \leavevmode%
  \hbox{%
  \begin{beamercolorbox}[wd=.333333\paperwidth,ht=2.25ex,dp=1ex,center]{author in head/foot}%
    \usebeamerfont{author in head/foot}\insertshortauthor
  \end{beamercolorbox}%
  \begin{beamercolorbox}[wd=.333333\paperwidth,ht=2.25ex,dp=1ex,center]{title in head/foot}%
    \usebeamerfont{title in head/foot}\insertshorttitle
  \end{beamercolorbox}%
  \begin{beamercolorbox}[wd=.333333\paperwidth,ht=2.25ex,dp=1ex,right]{date in head/foot}%
    \usebeamerfont{date in head/foot}\insertshortdate{}\hspace*{2em}
    \insertframenumber{} / \inserttotalframenumber\hspace*{2ex} 
  \end{beamercolorbox}}%
  \vskip0pt%
}

% ==============================================================
% Title Page Info
% ==============================================================
\title[Opt. Theory Mid-Term 3]{Optimization Theory And Application\\Mid-Term Presentation 3}
\subtitle{Part 1: Homework Q\&A\\Part 2: PRIME (ADMM-Adam Application)}
\author[TEAM 5]{Min-Tso Ko RE6144019\\ Yu-Jou Hsiao RE6141045\\Cheng-Jun Kang RE6144051}
\institute{Dept. of Data Science, NCKU}
\date{December 15, 2025}

\begin{document}

% Slide 1: Title
\frame{\titlepage}

% Slide 2: Outline
\begin{frame}
\frametitle{Outline}
\tableofcontents
\end{frame}

% ==============================================================
% PART 1: Homework
% ==============================================================
\section{Part I: Homework Q\&A}

\begin{frame}
\frametitle{Q1-Q4: Deep Learning  vs. Convex Optimization }

\begin{table}[h]
\centering
\renewcommand{\arraystretch}{1.75}
\begin{tabular}{l|p{0.3\textwidth}|p{0.3\textwidth}}
\toprule
& \textbf{Advantage (Pros)} & \textbf{Disadvantage (Cons)} \\
\midrule
\textbf{Deep Learning (DL)} & 
\textbf{Q1:} No heavy math required. (Data-driven feature extraction) & 
\textbf{Q3:} Requires Big Data to generalize well. \\
\hline
\textbf{Convex Opt. (CO)} & 
\textbf{Q2:} Does not rely on Big Data. (Math-driven rigor) & 
\textbf{Q4:} Involves heavy mathematics and derivations. \\
\bottomrule
\end{tabular}
\end{table}

\end{frame}

\begin{frame}
\frametitle{Q5-Q6: The ADMM-Adam Solution}

\begin{block}{Q5: How does ADMM-Adam theory avoid using big data?}
\textbf{Ans:} Even with small data, it can still extract useful information/statistics by leveraging \textbf{Convex Geometry} (regularizers) to guide the optimization process.
\end{block}

\vspace{1em}

\begin{block}{Q6: How does ADMM-Adam theory avoid using heavy math?}
\textbf{Ans:} It extracts features to design a simple convex regularizer. By using a \textbf{Deep Network (Adam)} to handle the complex mapping, we avoid the need to derive complex mathematical closed-form solutions.
\end{block}

\end{frame}

% ==============================================================
% PART 2: PRIME (Paper Summary)

% ==============================================================
\section{Part II: PRIME (ADMM-Adam Application)}

\begin{frame}
\frametitle{Paper Overview}

\textbf{Title:} PRIME: Unsupervised Multispectral Unmixing Using Virtual Quantum Prism and Convex Geometry 

\vspace{1em}

\textbf{Problem Statement: The "Underdetermined" Challenge}
\begin{itemize}
    \item \textbf{Multispectral Images (MSI):} Often have limited spatial resolution (Mixed-Pixel Phenomenon).
    \item \textbf{The Gap:} Standard Unmixing requires Bands ($P$) $>$ Sources ($N$).
    \item \textbf{Reality:} Satellites often have few bands (e.g., $P=4$) but many materials (e.g., $N=6$).
    \item \textbf{Current State:} Methods like VCA and NMF fail when $P < N$ .
\end{itemize}

\end{frame}

\begin{frame}
\frametitle{Proposed Solution: PRIME}

\begin{columns}
\column{0.7\textwidth}
\textbf{Core Innovation: Virtual Quantum Prism ($f$)}
\begin{itemize}
    \item PRIME inserts a virtual prism to split the limited MSI ($Z_m$) into a virtual Hyperspectral Image ($Z_h$).
    \item \textbf{Note:} This ``virtual HSI'' does not correspond to any real 
      HSI sensor—it's a mathematically constructed image to enable unmixing.
    \item \textbf{Transformation:} 
    $$ P < N \xrightarrow{\text{Prism } f} M > N $$
    \item Turns an unsolvable underdetermined problem into a solvable Virtual HU problem.
\end{itemize}

\column{0.3\textwidth}
\centering
% \includegraphics[width=\textwidth]{prism_diagram.png} 
% Placeholder for Prism concept (Fig 1 or 2 in paper)
\begin{beamercolorbox}[wd=\linewidth,ht=4cm,sep=1em,center,rounded=true,shadow=true]{block body example}
\footnotesize
\textbf{Concept:} \\
$Z_m$ (4 bands) \\
$\downarrow$ \\
\textbf{Prism $f$} \\
$\downarrow$ \\
$Z_h$ (8+ bands) \\
$\downarrow$ \\
Unmixing ($A, S$)
\end{beamercolorbox}
\end{columns}

\end{frame}

\begin{frame}
\frametitle{The CODE Framework (Methodology)}

PRIME is a perfect example of \textbf{CODE (Convex Optimization + Deep Learning)}.

\vspace{1em}

\textbf{1. Deep Learning Part (Adam):}
\begin{itemize}
    \item \textbf{Model:} QUEEN (Quantum Deep Network) acts as the function $f$.
    \item \textbf{Role:} Learns the "Light Splitting" physics to generate the virtual HSI ($Z_h$).
\end{itemize}

\vspace{0.5em}

\textbf{2. Convex Optimization Part (ADMM/Geometry):}
\begin{itemize}
    \item \textbf{Model:} HyperCSI (Hyperplane-based Craig Simplex Identification).
    \item \textbf{Role:} Solves the unmixing problem on $Z_h$ using Minimum Volume ($V(A)$) and Sparsity ($||S||_1$) regularization .
\end{itemize}

\end{frame}

\begin{frame}
\frametitle{Optimization Algorithm}

The objective function combines Data Fitting (DL) and Geometry Regularization (CO):

\begin{equation*}
\min_{A, S, Z_h, f} \underbrace{DF(A, S, Z_h, f)}_{\text{\textcolor{red}{Network Learning}}} + \lambda \underbrace{(V(A) + ||S||_1 + REG_f)}_{\text{\textcolor{blue}{Convex Regularization}}}
\end{equation*}

\vspace{1em}

\textbf{Alternating Optimization Strategy (Algorithm 1):}
\begin{enumerate}
    \item \textbf{Update $f$ (Prism):} Using \textbf{Adam} optimizer to minimize fitting error.
    \item \textbf{Update $Z_h$ (Virtual HSI):} Projecting onto non-negative space.
    \item \textbf{Update $A, S$ (Unmixing):} Using \textbf{HyperCSI} (Fast Convex Geometry method).
\end{enumerate}

\end{frame}

\begin{frame}
\frametitle{Experimental Results (Quantitative)}

\textbf{Scenario:} Unmixing $N=6$ sources from only $P=4$ bands (Underdetermined).

\begin{table}[h]
\centering
\small
\begin{tabular}{llcc}
\toprule
\textbf{Dataset} & \textbf{Method} & \textbf{SAM ($\downarrow$)} & \textbf{RMSE ($\downarrow$)} \\
\midrule
\multirow{3}{*}{Vancouver Island} & VCA & 13.32 & 0.30 \\
 & NMF & 12.89 & 0.24 \\
 & \textbf{PRIME (Ours)} & \textbf{6.54} & \textbf{0.16} \\
\midrule
\multirow{3}{*}{Montrose} & VCA & 3.85 & 0.16 \\
 & NMF & 4.13 & 0.20 \\
 & \textbf{PRIME (Ours)} & \textbf{2.21} & \textbf{0.09} \\
\bottomrule
\end{tabular}
\caption{Comparison with state-of-the-art HU methods adapted for MU.}
\end{table}

\end{frame}

\begin{frame}
\frametitle{Experimental Results (Visual)}

\begin{columns}
\column{0.4\textwidth}
\begin{itemize}
    \item \textbf{VCA \& NMF:} Fail to extract correct signatures; Abundance maps are noisy or empty (due to $P < N$).
    \item \textbf{PRIME:} Successfully separates 6 sources from 4 bands. The abundance maps recover clear spatial details matching the Ground Truth (GT).
\end{itemize}

\column{0.5\textwidth}
\centering
\vspace*{-1cm}
\includegraphics[width=0.6\linewidth]{Opt.3.jpg} \\

\end{columns}

\end{frame}

\begin{frame}
\frametitle{Conclusion}

\begin{block}{Summary}
PRIME is the first attempt to resolve the \textbf{Underdetermined ($P < N$)} Multispectral Unmixing problem.
\end{block}

\vspace{1em}

\textbf{Key Takeaways:}
\begin{itemize}
    \item \textbf{Virtual Prism:} Breaks the physical sensor limitations by generating virtual hyperspectral data.
    \item \textbf{CODE Theory:} Successfully blends the flexibility of Deep Learning (Adam) with the theoretical guarantees of Convex Geometry.
    \item \textbf{Impact:} Enables accurate material identification even with low-spectral-resolution sensors.
\end{itemize}

\end{frame}

\begin{frame}
\vfill
\begin{center}
\Large \textbf{Thank You for your attention!}
\vspace{0.3em}
\\
\normalsize All questions are welcome
\end{center}
\end{frame}
\end{document}