\documentclass{beamer}
\usetheme{AnnArbor}
\usecolortheme{spruce}
\usepackage{amsmath}
\usepackage{amssymb}
\usepackage{graphicx}
\usepackage{booktabs}

% Custom headline to show all sections
\setbeamertemplate{headline}
{
  \leavevmode%
  \hbox{%
  \begin{beamercolorbox}[wd=\paperwidth,ht=3.5ex,dp=1.5ex]{section in head/foot}%
    \usebeamerfont{section in head/foot}%
    \fontsize{8pt}{10pt}\selectfont%
    \insertsectionnavigationhorizontal{\paperwidth}{}{\hskip0pt plus1fill}{\hskip0pt plus1filll}
  \end{beamercolorbox}%
  }
}

% Footer with section progress
\setbeamertemplate{footline}
{
  \leavevmode%
  \hbox{%
  \begin{beamercolorbox}[wd=.333333\paperwidth,ht=2.25ex,dp=1ex,center]{author in head/foot}%
    \usebeamerfont{author in head/foot}\insertshortauthor
  \end{beamercolorbox}%
  \begin{beamercolorbox}[wd=.333333\paperwidth,ht=2.25ex,dp=1ex,center]{title in head/foot}%
    \usebeamerfont{title in head/foot}\insertshorttitle
  \end{beamercolorbox}%
  \begin{beamercolorbox}[wd=.333333\paperwidth,ht=2.25ex,dp=1ex,right]{date in head/foot}%
    \usebeamerfont{date in head/foot}\insertshortdate{}\hspace*{2em}
    \insertframenumber{} / \inserttotalframenumber\hspace*{2ex} 
  \end{beamercolorbox}}%
  \vskip0pt%
}

\title[AdamW]{Decoupled Weight Decay Regularization}
\subtitle{Loshchilov \& Hutter (ICLR 2019)}
\author[TEAM 5]{Min-Tso Ko RE6144019\\ Yu-Jou Hsiao RE6141045\\Cheng-Jun Kang RE6144051}
\institute{DS @ NCKU}
% \date{\today}
\date{November 10, 2025}

\begin{document}

\frame{\titlepage}

% ========== Section 1: Problem ==========
\section{The Problem}

\begin{frame}
\frametitle{The Problem with Adam}

\textbf{Key Issue:} L2 regularization $\neq$ Weight decay in Adam

\vspace{1em}

\begin{columns}[T]
    \begin{column}{0.48\textwidth}
        \textbf{SGD:} Equivalent
        \begin{itemize}
            \item $\mathcal{L}(\theta) = f(\theta) + \frac{\lambda}{2}\|\theta\|^2$
            \item $g_t = \nabla f(\theta_t) + \lambda \theta_t$
            \item Update:
            $\theta_{t+1} =  $\theta_{t} - \alpha g_t$
            $\theta_{t+1} = (1-\alpha\lambda)\theta_t - \alpha \nabla f(\theta_t)$
            \item Same as weight decay!
        \end{itemize}
    \end{column}
    \begin{column}{0.48\textwidth}
        \textbf{Adam:} NOT Equivalent
        \begin{itemize}
            \item $m_t = \beta_1 m_{t-1} + (1-\beta_1)g_t$
            \item $v_t = \beta_2 v_{t-1} + (1-\beta_2)g_t^2$
            \item $\hat{m}_t = m_t/(1-\beta_1^t)$
            \item $\hat{v}_t = v_t/(1-\beta_2^t)$
            \item Update:
            $\theta_{t+1} = \theta_t - \alpha \frac{\hat{m}_t}{\sqrt{\hat{v}_t}+\epsilon}$
            \item Large gradients\\
            $\rightarrow$ less regularization\\
            $\rightarrow$ Poor generalization
        \end{itemize}
    \end{column}
\end{columns}

% \vspace{1.5em}

\begin{block}{Result}
Adam with L2 regularization performs worse than SGD on image classification tasks
\end{block}

\end{frame}

% ========== Section 2: Solution ==========
\section{AdamW Solution}

\begin{frame}
\frametitle{The Solution: AdamW}

\textbf{Core Idea:} Decouple weight decay from gradient-based update

\vspace{1em}

\begin{columns}[T]
    \begin{column}{0.48\textwidth}
        \textbf{Adam}
        \begin{enumerate}
            \item $g_t = \nabla f(\theta_t) + \lambda \theta_t$
            \item $m_t, v_t$ updates
            \item $\theta_{t+1} = \theta_t - \alpha \frac{\hat{m}_t}{\sqrt{\hat{v}_t}+\epsilon}$
        \end{enumerate}
    \end{column}
    \begin{column}{0.48\textwidth}
        \textbf{AdamW}
        \begin{enumerate}
            \item $g_t = \nabla f(\theta_t)$ \textcolor{blue}{(no $\lambda\theta_t$)}
            \item $m_t, v_t$ updates
            \item $\theta_{t+1} = \theta_t - \alpha \frac{\hat{m}_t}{\sqrt{\hat{v}_t}+\epsilon}$ \textcolor{blue}{$- \lambda\theta_t$}
        \end{enumerate}
    \end{column}
\end{columns}

\vspace{1.5em}

\begin{block}{Benefits}
\begin{itemize}
    \item Weight decay applied \textbf{uniformly} to all parameters
    \item Decouples learning rate $\alpha$ and weight decay $\lambda$
    \item Simple fix: just one line of code!
\end{itemize}
\end{block}

\end{frame}

\begin{frame}
\frametitle{Theoretical Justification}

\textbf{Proposition 1:} For adaptive gradient methods, L2 reg $\neq$ weight decay

\vspace{0.5em}

\textbf{Proof sketch:}
\begin{itemize}
    \item Adam update: $\theta_{t+1} = \theta_t - \alpha M_t g_t$ where $M_t \neq kI$
    \item For equivalence, need: $\alpha M_t \lambda' \theta_t = \lambda \theta_t$ for all $\theta_t$
    \item Requires $M_t = kI$ (constant $\times$ identity), but Adam's $M_t$ is adaptive!
    \item Therefore, no L2 coefficient $\lambda'$ can make them equivalent
\end{itemize}

\vspace{1em}

\textbf{Proposition 2:} For fixed preconditioner, decoupled weight decay $\equiv$ scale-adjusted L2:
\[
f^{sreg}_t(\theta) = f_t(\theta) + \frac{\lambda'}{2\alpha} \|\theta \odot \sqrt{s}\|_2^2
\]

This means stronger regularization on parameters with large historical gradients!

\end{frame}

% ========== Section 3: Key Experiments ==========
\section{Experiments}

\begin{frame}
\frametitle{Experiment 1: Performance Comparison }

\textbf{Setup:} 26 2x64d ResNet on CIFAR-10, 100 epochs



\begin{figure}
    \centering
    \includegraphics[width=0.95\textwidth,height=0.55\textheight,keepaspectratio]{figure1.jpg}
    \caption{\footnotesize Top row: Adam (L2). Bottom row: AdamW. Darker = better performance. AdamW consistently achieves lower test error across all learning rate schedules.}
\end{figure}



{\footnotesize \textbf{Key Results:} AdamW outperforms Adam in all schedules with \textbf{15\% improvement}.}

\end{frame}

\begin{frame}
\frametitle{Experiment 2: Hyperparameter Decoupling }

\textbf{Goal:} Verify that $\alpha$ and $\lambda$ are decoupled in AdamW



\begin{figure}
    \centering
    \includegraphics[width=0.95\textwidth,height=0.55\textheight,keepaspectratio]{figure2.jpg}
    \caption{\footnotesize Heatmaps of test error for different $(\alpha, \lambda)$ combinations. Diagonal pattern = coupled. Grid pattern = decoupled. Top: SGD vs SGDW. Bottom: Adam vs AdamW.}
\end{figure}



{\footnotesize \textbf{Result:} AdamW shows grid pattern $\rightarrow$ easier hyperparameter tuning!}

\end{frame}

\begin{frame}
\frametitle{Experiment 3: Long Training \& Generalization }

\textbf{Setup:} 26 2x96d ResNet on CIFAR-10, 1800 epochs



\begin{figure}
    \centering
    \includegraphics[width=0.95\textwidth,height=0.55\textheight,keepaspectratio]{figure3.jpg}
    \caption{\footnotesize Top left: Training loss curves. Top right: Test error curves. Bottom: Test error vs. training loss scatter plot. AdamW (blue) shows better convergence and generalization than Adam (red).}
\end{figure}

\vspace{-1em}

{\footnotesize \textbf{Key Finding:} At the same training loss, AdamW achieves lower test error.}

\end{frame}

% ========== Section 4: Impact ==========
\section{Impact \& Summary}

\begin{frame}
\frametitle{Additional Contributions}

\textbf{1. Normalized Weight Decay}
\begin{itemize}
    \item Problem: Optimal $\lambda$ depends on training budget
    \item Solution: $\lambda = \lambda_{norm}\sqrt{\frac{b}{BT}}$ where $b$=batch size, $B$=dataset size, $T$=epochs
    \item Makes hyperparameters more robust across different training lengths
\end{itemize}

\vspace{1em}

\textbf{2. Warm Restarts (AdamWR)}
\begin{itemize}
    \item Combine AdamW with cosine annealing + periodic restarts
    \item Improves anytime performance by up to \textbf{10x speedup}
    \item Achieves same final accuracy in fewer epochs
\end{itemize}

\end{frame}

\begin{frame}
\frametitle{Impact \& Adoption}

\textbf{Widely Adopted in Practice:}
\begin{itemize}
    \item \textbf{PyTorch:} \texttt{torch.optim.AdamW}
    \item \textbf{TensorFlow:} \texttt{tf.keras.optimizers.AdamW}
    \item \textbf{Hugging Face:} Default optimizer for BERT, GPT, etc.
    \item Used in training most modern large language models
\end{itemize}

\vspace{1em}

\textbf{Real-World Applications:}
\begin{itemize}
    \item Face detection (Wang et al., 2018): 10x faster with same accuracy
    \item EEG classification (Völker et al., 2018): 72\% vs 61\% accuracy
    \item NLP (Radford et al., 2018): State-of-the-art on multiple benchmarks
\end{itemize}

\vspace{1em}

\textbf{Why Important?}
\begin{itemize}
    \item Simple fix (1 line) with major impact (15\% improvement)
    \item Makes Adam competitive with SGD on vision tasks
    \item Simplifies hyperparameter tuning
\end{itemize}

\end{frame}

\begin{frame}
\frametitle{Summary}

\begin{block}{Problem}
L2 regularization in Adam is incorrectly implemented and doesn't work as intended
\end{block}

\begin{block}{Solution: AdamW}
Decouple weight decay from gradient-based update
\[
\theta_{t+1} = \theta_t - \alpha \frac{\hat{m}_t}{\sqrt{\hat{v}_t}+\epsilon} - \lambda\theta_t
\]
\end{block}

\begin{block}{Results}
\begin{itemize}
    \item 15\% improvement in test error
    \item Decoupled hyperparameters (easier tuning)
    \item Better generalization than Adam with L2
    \item Competitive with SGD on image classification
\end{itemize}
\end{block}

\begin{block}{Impact}
Now the standard optimizer for training Transformers and LLMs
\end{block}

\end{frame}

\begin{frame}
\frametitle{Key Takeaways}

\begin{enumerate}
    \item \textbf{Simple but overlooked:} L2 reg = weight decay in SGD, but NOT in Adam
    
    \item \textbf{One-line fix:} Just decouple the weight decay step
    
    \item \textbf{Major improvement:} 15\% better generalization
    
    \item \textbf{Easier tuning:} Decoupled hyperparameters
    
    \item \textbf{Wide adoption:} Standard for modern deep learning
\end{enumerate}

\vspace{2em}

\begin{center}
\textbf{Lesson:} Small implementation details can have huge impacts!
\end{center}

\end{frame}

\begin{frame}
\frametitle{References}

\begin{itemize}
    \item \textbf{Loshchilov, I., \& Hutter, F. (2019).} Decoupled weight decay regularization. ICLR.
    
    \item Kingma, D. P., \& Ba, J. (2015). Adam: A method for stochastic optimization. ICLR.
    
    \item Loshchilov, I., \& Hutter, F. (2016). SGDR: Stochastic gradient descent with warm restarts. ICLR.
    
    \item Zhang, G., et al. (2018). Three mechanisms of weight decay regularization. arXiv preprint.
\end{itemize}

\vfill
\begin{center}
\Large \textbf{Thank You for your attention!}
\vspace{0.3em}
\\
\normalsize All questions are welcome
\end{center}
\end{frame}

% ========== Backup Slides ==========
\appendix

\begin{frame}
\frametitle{Backup: Algorithm Comparison}

\begin{table}[h]
\centering
\small
\begin{tabular}{p{0.45\textwidth}p{0.45\textwidth}}
\toprule
\textbf{Adam (L2)} & \textbf{AdamW} \\
\midrule
$g_t = \nabla f + \textcolor{red}{\lambda \theta_t}$ & $g_t = \nabla f$ \\
$m_t = \beta_1 m_{t-1} + (1-\beta_1)g_t$ & $m_t = \beta_1 m_{t-1} + (1-\beta_1)g_t$ \\
$v_t = \beta_2 v_{t-1} + (1-\beta_2)g_t^2$ & $v_t = \beta_2 v_{t-1} + (1-\beta_2)g_t^2$ \\
$\hat{m}_t = m_t/(1-\beta_1^t)$ & $\hat{m}_t = m_t/(1-\beta_1^t)$ \\
$\hat{v}_t = v_t/(1-\beta_2^t)$ & $\hat{v}_t = v_t/(1-\beta_2^t)$ \\
$\theta_t = \theta_{t-1} - \alpha\frac{\hat{m}_t}{\sqrt{\hat{v}_t}+\epsilon}$ & $\theta_t = \theta_{t-1} - \alpha\frac{\hat{m}_t}{\sqrt{\hat{v}_t}+\epsilon}$ \textcolor{blue}{$- \lambda\theta_{t-1}$} \\
\bottomrule
\end{tabular}
\end{table}

\textbf{Key difference:} Weight decay is applied \textit{after} adaptive scaling in AdamW

\end{frame}

\begin{frame}
\frametitle{Backup: Recommended Hyperparameters}

\begin{table}[h]
\centering
\begin{tabular}{lcc}
\toprule
\textbf{Task} & \textbf{Learning Rate $\alpha$} & \textbf{Weight Decay $\lambda_{norm}$} \\
\midrule
Image (small) & 0.001 & 0.025 - 0.05 \\
Image (large) & 0.001 & 0.01 - 0.025 \\
NLP (Transformer) & 0.0001 - 0.0003 & 0.01 \\
Generative & 0.0001 - 0.001 & 0.0 - 0.01 \\
\bottomrule
\end{tabular}
\end{table}

\vspace{1em}

\textbf{General advice:}
\begin{itemize}
    \item Start with $\alpha=0.001$, $\lambda_{norm}=0.025$
    \item Use cosine annealing or warm restarts
    \item Tune $\alpha$ and $\lambda$ independently (they're decoupled!)
\end{itemize}

\end{frame}

\end{document}