\documentclass{beamer}
\usetheme{AnnArbor}
\usecolortheme{spruce}
\usepackage{amsmath}
\usepackage{amssymb}
\usepackage{graphicx}
\usepackage{booktabs} % 用於更好看的表格線條

% 自定義 headline 顯示所有章節
\setbeamertemplate{headline}
{
  \leavevmode%
  \hbox{%
  \begin{beamercolorbox}[wd=\paperwidth,ht=3.5ex,dp=1.5ex]{section in head/foot}%
    \usebeamerfont{section in head/foot}% 使用較小字體
    \fontsize{8pt}{10pt}\selectfont% 設定字體大小為 8pt
    \insertsectionnavigationhorizontal{\paperwidth}{}{\hskip0pt plus1fill}{\hskip0pt plus1filll}
  \end{beamercolorbox}%
  }
}

% Footer with section progress
\setbeamertemplate{footline}
{
  \leavevmode%
  \hbox{%
  \begin{beamercolorbox}[wd=.333333\paperwidth,ht=2.25ex,dp=1ex,center]{author in head/foot}%
    \usebeamerfont{author in head/foot}\insertshortauthor
  \end{beamercolorbox}%
  \begin{beamercolorbox}[wd=.333333\paperwidth,ht=2.25ex,dp=1ex,center]{title in head/foot}%
    \usebeamerfont{title in head/foot}\insertshorttitle
  \end{beamercolorbox}%
  \begin{beamercolorbox}[wd=.333333\paperwidth,ht=2.25ex,dp=1ex,right]{date in head/foot}%
    \usebeamerfont{date in head/foot}\insertshortdate{}\hspace*{2em}
    \insertframenumber{} / \inserttotalframenumber\hspace*{2ex} 
  \end{beamercolorbox}}%
  \vskip0pt%
}

\title{SGD with Momentum (SGDM)}
\author[TEAM 5]{Min-Tso Kov RE6144019\\ Yu-Jou Hsiao RE6141045\\Cheng-Jun Kang RE6144051 }
\institute{DS @ NCKU}
\date{\today} 

\begin{document}

\frame{\titlepage}

% Table of Contents
\begin{frame}
\frametitle{Outline}
\tableofcontents
\end{frame}

% ========== Section 1 ==========
\section{Background}

\begin{frame}
\frametitle{Background and Motivation}

\textbf{Problems with Standard SGD}

Standard Stochastic Gradient Descent update rule:
\[
\theta_{t+1} = \theta_t - \alpha \nabla f(\theta_t)
\]

\begin{columns}[T] % 使用 columns 環境來並排文字和圖片
    \begin{column}{0.5\textwidth}
        \textbf{Main Issues:}
        \begin{itemize}
            \item \textbf{Slow Convergence}
            \item \textbf{Oscillation} in ravines
            \item \textbf{Sensitive Learning Rate}
        \end{itemize}
        \vspace{1em}
        % \textbf{Motivation:} Can we use momentum to accelerate optimization?
    \end{column}
    \begin{column}{0.5\textwidth}
        % <-- 已新增:建議在此處插入 SGD 震盪的視覺化圖片
        \begin{figure}
            \includegraphics[width=0.6\linewidth]{SGD.jpg}
            \caption{\footnotesize SGD often oscillates in narrow valleys.}
        \end{figure}
    \end{column}
\end{columns}

\end{frame}

% ========== Section 2 ==========
\section{Intuition}

\begin{frame}
\frametitle{Physical Intuition}

\textbf{Analogy: A Ball Rolling Down a Hill}

\begin{columns}[T]
    \begin{column}{0.4\textwidth}
        % <-- 已新增:建議在此處插入小球滾下山的示意圖
        \begin{figure}
            \includegraphics[width=\linewidth]{ball_rolling.png} % <-- 請替換成您的圖片檔名
        \end{figure}
    \end{column}
    \begin{column}{0.6\textwidth}
        \begin{center}
        \begin{tabular}{c|c}
        \textbf{Physical Concept} & \textbf{Optimization} \\ \hline
        Position & Parameter $\theta$ \\
        Velocity & Momentum $v$ \\
        Gravity (Slope) & Gradient $\nabla f$ \\
        Friction & Momentum $\mu$
        \end{tabular}
        \end{center}
        \vspace{1em}
        \textbf{Key Idea:} Consider not only the current gradient, but also the historical direction.
    \end{column}
\end{columns}

\end{frame}

% ========== Section 3 ==========
\section{Derivation}

\begin{frame}
\frametitle{Mathematical Derivation}

\textbf{Momentum SGD Algorithm}

\textbf{Initialization:}
\[
v_0 = 0 \quad \text{(Initial velocity is zero)}
\]

\textbf{Update at Each Iteration ($t$):}

\textbf{Step 1:} Compute Gradient
\[
g_t = \nabla f(\theta_t)
\]

% <-- 已修改:更新為更常見的公式表達
\textbf{Step 2:} Update Momentum (Accumulate Gradients)
\[
v_{t+1} = \mu \cdot v_t + g_t 
\]

\textbf{Step 3:} Update Parameters (Apply Momentum with LR)
\[
\theta_{t+1} = \theta_t - \alpha \cdot v_{t+1}
\]

\vspace{0.5em}

\textbf{Hyperparameters:}
\begin{itemize}
    \item $\alpha$: Learning rate, typically $0.01 \sim 0.1$
    \item $\mu$: Momentum coefficient, typically $0.9 \sim 0.99$
\end{itemize}

\end{frame}

% ========== Section 4 ==========
\section{Properties}

\begin{frame}
\frametitle{Mathematical Properties Analysis}

\textbf{Understanding through Expansion}

Expanding the parameter update term:
\begin{align*}
\theta_{t+1} &= \theta_t - \alpha v_{t+1} \\
&= \theta_t - \alpha (\mu v_t + g_t) \\
&= \theta_t - \alpha (\mu (\mu v_{t-1} + g_{t-1}) + g_t) \\
&= \theta_t - \alpha \sum_{i=0}^{t} \mu^{t-i} g_i
\end{align*}

\vspace{0.5em}

\textbf{Key Insights:}
\begin{itemize}
    \item The current step considers a \textbf{weighted sum of all historical gradients}.
    \item More recent gradients have larger weights (exponential decay).
    \item Consistent gradient direction $\rightarrow$ Velocity accumulation, faster convergence.
    % \item Oscillating gradient direction $\rightarrow$ Mutual cancellation, reduced oscillation.
\end{itemize}

\end{frame}

\begin{frame}
\frametitle{Mathematical Properties (Continued)}

\textbf{Effective Learning Rate}

When gradients are stable ($g_t \approx g$), the steady-state velocity update:
\[
\Delta \theta \approx - \alpha \left( \frac{1}{1-\mu} \right) g
\]

\textbf{Practical Effect:} The step size is amplified by a factor of $\displaystyle \frac{1}{1-\mu}$.

\vspace{1em}

\textbf{Example:} When $\mu = 0.9$
\[
\text{Effective Step Size} = \frac{1}{1-0.9} = 10
\]
The step size is amplified by \textbf{10 times}!

\end{frame}

% ========== Section 5 ==========
\section{Effects}

\begin{frame}
\frametitle{Effects and Advantages}

\begin{columns}[T]
    \begin{column}{0.5\textwidth}
        \textbf{Three Major Advantages}
        \begin{enumerate}
            \item \textbf{Accelerated Convergence}
            \begin{itemize}
                \item Accumulates velocity
                \item Faster progress
            \end{itemize}
            
            \item \textbf{Reduced Oscillation}
            \begin{itemize}
                \item Cancels opposing gradients
                \item More stable training
            \end{itemize}
            
            \item \textbf{Escape Saddle Points}
            \begin{itemize}
                \item Inertia helps pass flat regions
            \end{itemize}
        \end{enumerate}
    \end{column}
    \begin{column}{0.5\textwidth}
        % <-- 已新增:建議在此處插入 SGD vs SGDM 的收斂路徑比較圖
        \begin{figure}
            \includegraphics[width=\linewidth]{sgd_vs_sgdm.jpg} % <-- 請替換成您的圖片檔名
            \caption{\footnotesize SGDM (blue) converges faster and more smoothly than SGD (red).}
        \end{figure}
    \end{column}
\end{columns}

\end{frame}

% ========== Section 6 ==========
\section{Impact}

\begin{frame}
\frametitle{Evolution of Optimizers}

\begin{center}
\footnotesize
\begin{tabular}{l}
SGD (1951) \\
\quad $\downarrow$ \\
\textbf{SGD + Momentum (1999)} $\leftarrow$ This Paper's Focus \\
\quad $\downarrow$ \\
\quad $\bullet$ Nesterov Accelerated Gradient (NAG, 2013) \\
\quad \quad $\rightarrow$ NAdam (2016) \\
\quad $\bullet$ Adam (2015) - Combines \textbf{Momentum} + RMSprop \\
\quad \quad $\rightarrow$ AdamW (2019) \\
\quad \quad $\rightarrow$ RAdam (2019) \\
\end{tabular}
\end{center}

\vspace{1em}
\textbf{Core Position:} Momentum is a cornerstone of modern optimizers.

\end{frame}

% <-- 已新增:SGDM vs Adam 的比較表格
\begin{frame}
\frametitle{Comparison: SGDM vs. Adam}

\begin{tabular}{l p{4.0cm} p{4.0cm}}
\toprule
\textbf{Feature} & \textbf{SGD with Momentum (SGDM)} & \textbf{Adam} \\
\midrule
\textbf{Core Idea} & First-order momentum (historical gradients) & First-order momentum \textbf{and} Second-order momentum (historical squared gradients) \\
\addlinespace
\textbf{Learning Rate} & Global and fixed & Per-parameter adaptive learning rate \\
\addlinespace
\textbf{Pros} & Better generalization, stable in later stages & Fast initial convergence, less sensitive to LR choice \\
\addlinespace
\textbf{Cons} & Requires careful LR tuning & May converge to poorer local optima, higher memory usage \\
\addlinespace
\textbf{Best For} & Computer Vision (CNNs) & NLP (Transformers), Fast Prototyping \\
\bottomrule
\end{tabular}

\end{frame}


\begin{frame}
\frametitle{Applications in Computer Vision}

\textbf{Practical Application Cases}

\textbf{Image Classification:}
\begin{itemize}
    \item ResNet, VGG training commonly uses SGD + Momentum
    \item Hyperparameters: $\mu = 0.9$, $\alpha = 0.1$ (with decay)
\end{itemize}

\textbf{Object Detection / Segmentation:}
\begin{itemize}
    \item Faster R-CNN, YOLO, U-Net series
    \item Momentum helps stabilize training, especially with large batches
\end{itemize}

\vspace{1em}
\textbf{Current Trends:}
\begin{itemize}
    \item \textbf{CNN-based Models:} Still often prefer \textbf{SGDM} for better final performance.
    \item \textbf{Transformer-based Models:} Almost always use \textbf{Adam} or its variants (AdamW).
\end{itemize}

\end{frame}

% ========== Section 7 ==========
\section{Summary}

\begin{frame}
\frametitle{Summary}

\textbf{Core Contributions}
\begin{enumerate}
    \item \textbf{Introduced Physical Intuition} into optimization
    \item \textbf{Simple and Effective}, adding only one hyperparameter ($\mu$)
    \item \textbf{Established a Foundation} for modern optimizers like Adam
\end{enumerate}

\vspace{1em}
% <-- 已修改:更新為新的公式
\textbf{Key Formulas Recap}
\[
v_{t+1} = \mu \cdot v_t + g_t
\]
\[
\theta_{t+1} = \theta_t - \alpha \cdot v_{t+1}
\]

\vspace{1em}
\textbf{Momentum is not just a trick, but a fundamental concept in deep learning optimization.}

\end{frame}

\begin{frame}
\frametitle{References}
\begin{itemize}
    \item Qian, N. (1999). \textit{On the momentum term in gradient descent learning algorithms}. Neural Networks.
    
    \item Sutskever, I., et al. (2013). \textit{On the importance of initialization and momentum in deep learning}. ICML.
    
    \item Kingma, D. P., \& Ba, J. (2015). \textit{Adam: A method for stochastic optimization}. ICLR.
\end{itemize}

\vfill
\begin{center}
\Large \textbf{Thank you for your attention}
\vspace{0.5em}
\\
\normalsize Any Questions?
\end{center}
\end{frame}

\end{document}